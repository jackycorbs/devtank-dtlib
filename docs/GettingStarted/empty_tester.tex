\documentclass[a4paper,12pt, notitlepage]{article}

\usepackage[top=25mm,bottom=25mm,left=25mm,right=25mm]{geometry}
%\usepackage{amsmath} 
\usepackage{graphicx}
%\usepackage{epstopdf}
\usepackage{listings}
\usepackage{color}
\usepackage{url}
\usepackage{setspace} 
%\usepackage[square, numbers]{natbib}
\usepackage{titlesec}
\usepackage{fancyhdr}
\usepackage[ddmmyyyy]{datetime}

\setstretch{1.44}
\setlength{\columnsep}{6mm}
\titleformat{\section}{\bfseries\large\scshape\filright}{\thesection}{1em}{}
\titleformat{\subsection}{\bfseries\normalsize\scshape\filright}{\thesubsection}{1em}{}


\renewcommand{\abstractname}{}
\newcommand{\captionfonts}{\footnotesize}
\renewcommand\thesection{\arabic{section}.}
\renewcommand\thesubsection{\arabic{section}.\arabic{subsection}}

\makeatletter
\long\def\@makecaption#1#2{
  \vskip\abovecaptionskip
  \sbox\@tempboxa{{\captionfonts #1: #2}}
  \ifdim \wd\@tempboxa >\hsize
    {\captionfonts #1: #2\par}
  \else
    \hbox to\hsize{\hfil\box\@tempboxa\hfil}
  \fi
  \vskip\belowcaptionskip}
    
\makeatother

\definecolor{dkgreen}{rgb}{0,0.6,0}
\definecolor{gray}{rgb}{0.5,0.5,0.5}
\definecolor{mauve}{rgb}{0.58,0,0.82}

\lstset{frame=tb,
	language=python,
	aboveskip=3mm,
	belowskip=3mm,
	showstringspaces=false,
	columns=flexible,
	basicstyle={\small\ttfamily},
	numbers=none,
	numberstyle=\tiny\color{gray},
	keywordstyle=\color{blue},
	commentstyle=\color{dkgreen},
	stringstyle=\color{mauve},
	breaklines=true,
	breakatwhitespace=true,
	tabsize=3
}

\pagestyle{fancy}
\fancyhf{}
\rhead{Create Empty Production Tester}
\lhead{\includegraphics[height=1cm]{./Images/logo.pdf}}
\rfoot{Page \thepage}
\lfoot{}

\renewcommand{\dateseparator}{.}

\begin{document}
%%%%%%%%%%%%%%%%%%%%%%%%%%%%%%%%%%%%%%%%%%%%%%%%%%%%%%%%%%%%%%%%%%%%%%%%%%%%%%%%%%%%
\title{\textbf{\large{Create Empty Production Tester}}}

\author{\normalsize{Devtank Ltd.} \\
        \small\textit{
        Marcus Holder}}
\date{\today}

\maketitle 
\thispagestyle{fancy}

%%%%%%%%%%%%%%%%%%%%%%%%%%%%%%%%%%%%%%%%%%%%%%%%%%%%%%%%%%%%%%%%%%%%%%%%%%%%%%%%%%%%

\begin{abstract} 
\noindent
This is an overview of getting started with building your own production tester using an existing template.
\end{abstract}
\vspace{11mm}

\newpage
\tableofcontents
\newpage


%%%%%%%%%%%%%%%%%%%%%%%%%%%%%%%%%%%%%%%%%%%%%%%%%%%%%%%%%%%%%%%%%%%%%%%%%%%%%%%%%%%%
\section{Starting}
\label{sec: start}

\subsection{Run script.}
\label{renameStart}

To create an empty production tester, you will need to execute the \url{create_empty_prod_tester.sh} script. It is located in \url{/dtlib/docs/GettingStarted/}. It requires an argument which will be the name of your project, the script takes this name and creates a git repository of this name and applies the name to all of the applications in the template GUI and CLI.

It will also place \url{dtlib} inside your new project as a git submodule. To create a project called \url{first_prod_tester}, run \url{./create_empty_prod_tester.sh} \url{first_prod_tester}.

\subsection{Run GUI}

The git repository is located in your home directory. To run the production tester, type \url{cd} \url{~/first_prod_tester/first_prod_tester_gui/} and then run \url{./output/bin/first_prod_tester_tester_gui.sh} \url{--desktop} \url{--production}.

The only tests that will be run include a test to read the serial number and one to read the UID.
%%%%%%%%%%%%%%%%%%%%%%%%%%%%%%%%%%%%%%%%%%%%%%%%%%%%%%%%%%%%%%%%%%%%%%%%%%%%%%%%%%%%

\end{document}