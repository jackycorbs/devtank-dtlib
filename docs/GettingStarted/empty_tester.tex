\documentclass[a4paper,12pt, notitlepage]{article}

\usepackage[top=25mm,bottom=25mm,left=25mm,right=25mm]{geometry}
%\usepackage{amsmath} 
\usepackage{graphicx}
%\usepackage{epstopdf}
\usepackage{listings}
\usepackage{color}
\usepackage{url}
\usepackage{setspace} 
%\usepackage[square, numbers]{natbib}
\usepackage{titlesec}
\usepackage{fancyhdr}
\usepackage[ddmmyyyy]{datetime}

\setstretch{1.44}
\setlength{\columnsep}{6mm}
\titleformat{\section}{\bfseries\large\scshape\filright}{\thesection}{1em}{}
\titleformat{\subsection}{\bfseries\normalsize\scshape\filright}{\thesubsection}{1em}{}


\renewcommand{\abstractname}{}
\newcommand{\captionfonts}{\footnotesize}
\renewcommand\thesection{\arabic{section}.}
\renewcommand\thesubsection{\arabic{section}.\arabic{subsection}}

\makeatletter
\long\def\@makecaption#1#2{
  \vskip\abovecaptionskip
  \sbox\@tempboxa{{\captionfonts #1: #2}}
  \ifdim \wd\@tempboxa >\hsize
    {\captionfonts #1: #2\par}
  \else
    \hbox to\hsize{\hfil\box\@tempboxa\hfil}
  \fi
  \vskip\belowcaptionskip}
    
\makeatother

\definecolor{dkgreen}{rgb}{0,0.6,0}
\definecolor{gray}{rgb}{0.5,0.5,0.5}
\definecolor{mauve}{rgb}{0.58,0,0.82}

\lstset{frame=tb,
	language=python,
	aboveskip=3mm,
	belowskip=3mm,
	showstringspaces=false,
	columns=flexible,
	basicstyle={\small\ttfamily},
	numbers=none,
	numberstyle=\tiny\color{gray},
	keywordstyle=\color{blue},
	commentstyle=\color{dkgreen},
	stringstyle=\color{mauve},
	breaklines=true,
	breakatwhitespace=true,
	tabsize=3
}

\pagestyle{fancy}
\fancyhf{}
\rhead{Create Empty Production Tester}
\lhead{\includegraphics[height=1cm]{./Images/logo.pdf}}
\rfoot{Page \thepage}
\lfoot{}

\renewcommand{\dateseparator}{.}

\begin{document}
%%%%%%%%%%%%%%%%%%%%%%%%%%%%%%%%%%%%%%%%%%%%%%%%%%%%%%%%%%%%%%%%%%%%%%%%%%%%%%%%%%%%
\title{\textbf{\large{Create Empty Production Tester}}}

\author{\normalsize{Devtank Ltd.} \\
        \small\textit{
        Marcus Holder}}
\date{\today}

\maketitle 
\thispagestyle{fancy}

%%%%%%%%%%%%%%%%%%%%%%%%%%%%%%%%%%%%%%%%%%%%%%%%%%%%%%%%%%%%%%%%%%%%%%%%%%%%%%%%%%%%

\begin{abstract} 
\noindent
This is an overview of getting started with building your own production tester using an existing template.
\end{abstract}
\vspace{11mm}

\newpage
\tableofcontents
\newpage


%%%%%%%%%%%%%%%%%%%%%%%%%%%%%%%%%%%%%%%%%%%%%%%%%%%%%%%%%%%%%%%%%%%%%%%%%%%%%%%%%%%%
\section{Starting}
\label{sec: start}

\subsection{Run script.}
\label{renameStart}

To create an empty production tester, you will need to execute the \url{create_empty_prod_tester.sh} script. It is located in \url{/dtlib/docs/GettingStarted/}. Before running the script, move to the location where you want to store the production tester, e.g. \url{cd ~/Documents/my_prod_tester_work}, then run the script with \url{./$DTLIB/docs/GettingStarted/create_empty_prod_tester.sh} \url{my_prod_tester}. The argument \url{my_prod_tester} is the name of your empty production tester and the applications inside it. Note that \url{$DTLIB} defines your version of the dtlib git repository. The script will also place a fresh \url{dtlib} inside your new project as a git submodule. 

\subsection{Run GUI}

To run the production tester, enter the \url{apps} directory and then the template gui directory which will be located in \url{~/Documents/my_prod_tester_work/my_prod_tester_gui/} and then run \url{./output/bin/my_prod_tester_tester_gui.sh} \url{--desktop}.

Once the gui starts you will be asked for a barcode scan, as we are just in production mode you can enter a matching number. The --desktop argument if removed will open the gui in full screen mode. In the next window, select 'Yes' when asked if the board has been placed in the fixture and you will be able to see the one example test run.
%%%%%%%%%%%%%%%%%%%%%%%%%%%%%%%%%%%%%%%%%%%%%%%%%%%%%%%%%%%%%%%%%%%%%%%%%%%%%%%%%%%%

\end{document}