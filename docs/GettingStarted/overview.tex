\documentclass[a4paper,12pt, notitlepage]{article}

\usepackage[top=25mm,bottom=25mm,left=25mm,right=25mm]{geometry}
%\usepackage{amsmath} 
\usepackage{graphicx}
%\usepackage{epstopdf}
\usepackage{listings}
\usepackage{color}
\usepackage{url}
\usepackage{setspace} 
%\usepackage[square, numbers]{natbib}
\usepackage{titlesec}
\usepackage{fancyhdr}
\usepackage[ddmmyyyy]{datetime}

\setstretch{1.44}
\setlength{\columnsep}{6mm}
\titleformat{\section}{\bfseries\large\scshape\filright}{\thesection}{1em}{}
\titleformat{\subsection}{\bfseries\normalsize\scshape\filright}{\thesubsection}{1em}{}


\renewcommand{\abstractname}{}
\newcommand{\captionfonts}{\footnotesize}
\renewcommand\thesection{\arabic{section}.}
\renewcommand\thesubsection{\arabic{section}.\arabic{subsection}}

\makeatletter
\long\def\@makecaption#1#2{
  \vskip\abovecaptionskip
  \sbox\@tempboxa{{\captionfonts #1: #2}}
  \ifdim \wd\@tempboxa >\hsize
    {\captionfonts #1: #2\par}
  \else
    \hbox to\hsize{\hfil\box\@tempboxa\hfil}
  \fi
  \vskip\belowcaptionskip}
    
\makeatother

\definecolor{dkgreen}{rgb}{0,0.6,0}
\definecolor{gray}{rgb}{0.5,0.5,0.5}
\definecolor{mauve}{rgb}{0.58,0,0.82}

\lstset{frame=tb,
	language=python,
	aboveskip=3mm,
	belowskip=3mm,
	showstringspaces=false,
	columns=flexible,
	basicstyle={\small\ttfamily},
	numbers=none,
	numberstyle=\tiny\color{gray},
	keywordstyle=\color{blue},
	commentstyle=\color{dkgreen},
	stringstyle=\color{mauve},
	breaklines=true,
	breakatwhitespace=true,
	tabsize=3
}

\pagestyle{fancy}
\fancyhf{}
\rhead{Getting Started: Overview}
\lhead{\includegraphics[height=1cm]{./Images/logo.pdf}}
\rfoot{Page \thepage}
\lfoot{}

\renewcommand{\dateseparator}{.}

\begin{document}
%%%%%%%%%%%%%%%%%%%%%%%%%%%%%%%%%%%%%%%%%%%%%%%%%%%%%%%%%%%%%%%%%%%%%%%%%%%%%%%%%%%%
\title{\textbf{\large{Getting Started with the HILTOP: Overview}}}

\author{\normalsize{Devtank Ltd.} \\
        \small\textit{
        Harry Geyer}}
\date{\today}

\maketitle 
\thispagestyle{fancy}

%%%%%%%%%%%%%%%%%%%%%%%%%%%%%%%%%%%%%%%%%%%%%%%%%%%%%%%%%%%%%%%%%%%%%%%%%%%%%%%%%%%%

\begin{abstract} 
\noindent
This is an overview of getting started with the HILTOP. If more detailed required, use the more specialised documentation attached.
\end{abstract}
\vspace{11mm}

\newpage
\tableofcontents
\newpage

%%%%%%%%%%%%%%%%%%%%%%%%%%%%%%%%%%%%%%%%%%%%%%%%%%%%%%%%%%%%%%%%%%%%%%%%%%%%%%%%%%%%
\section{Prerequisites}
\label{sec: prereq}

\textbf{Hardware:}

\begin{enumerate}
  \item HILTOP
  \item Multimeter
  \item Linux machine to work with (Suggested)
\end{enumerate}

\noindent
\textbf{Software:}

\begin{enumerate}
  \item HILTOP development image
  \item libdevtankreborn library
\end{enumerate}

%%%%%%%%%%%%%%%%%%%%%%%%%%%%%%%%%%%%%%%%%%%%%%%%%%%%%%%%%%%%%%%%%%%%%%%%%%%%%%%%%%%%
\section{Starting}
\label{sec: start}

\subsection{Renaming}
\label{renameStart}

First to rename your files, directories and contents of files to your project name, use the bash script in the base directory: \lstinline!./rename.sh <your_project_name>!. Ensure name does cannot be interpretted as a function i.e. ``print" is a bad name. As long as every project name is unique, renaming can be repeated if a name change is needed. In these documents, it is assumed this step has not been taken so nomenclature will be simple: replace ``example" with your project name.

%%%%%%%%%%%%%%%%%%%%%%%%%%%%%%%%%%%%%%%%%%%%%%%%%%%%%%%%%%%%%%%%%%%%%%%%%%%%%%%%%%%%
\small{
\begin{thebibliography}{99}

\setlength{\itemsep}{-2mm}

\bibitem{Webpage} Page Title,
                  Author {\url{https://www.url.org/}} {Accessed: dd.mm.yy}.
\bibitem{Book} Author, 
                  {\em Book Name}. Publisher {\bf Edition}, Pages (Publish Year).

\end{thebibliography}
}

\end{document}